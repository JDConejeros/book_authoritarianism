\documentclass[11pt,]{article}
\usepackage{lmodern}
\usepackage{amssymb,amsmath}
\usepackage{ifxetex,ifluatex}
\usepackage{fixltx2e} % provides \textsubscript
\ifnum 0\ifxetex 1\fi\ifluatex 1\fi=0 % if pdftex
  \usepackage[T1]{fontenc}
  \usepackage[utf8]{inputenc}
\else % if luatex or xelatex
  \ifxetex
    \usepackage{mathspec}
  \else
    \usepackage{fontspec}
  \fi
  \defaultfontfeatures{Ligatures=TeX,Scale=MatchLowercase}
\fi
% use upquote if available, for straight quotes in verbatim environments
\IfFileExists{upquote.sty}{\usepackage{upquote}}{}
% use microtype if available
\IfFileExists{microtype.sty}{%
\usepackage{microtype}
\UseMicrotypeSet[protrusion]{basicmath} % disable protrusion for tt fonts
}{}
\usepackage[left=2cm,right=2cm,top=1.8cm,bottom=1.8cm]{geometry}
\usepackage{hyperref}
\PassOptionsToPackage{usenames,dvipsnames}{color} % color is loaded by hyperref
\hypersetup{unicode=true,
            pdfauthor={Daniel Miranda, Juan Carlos Castillo, Catalina Miranda \& José Daniel Conejeros.},
            colorlinks=true,
            linkcolor=blue,
            citecolor=Blue,
            urlcolor=blue,
            breaklinks=true}
\urlstyle{same}  % don't use monospace font for urls
\usepackage{graphicx,grffile}
\makeatletter
\def\maxwidth{\ifdim\Gin@nat@width>\linewidth\linewidth\else\Gin@nat@width\fi}
\def\maxheight{\ifdim\Gin@nat@height>\textheight\textheight\else\Gin@nat@height\fi}
\makeatother
% Scale images if necessary, so that they will not overflow the page
% margins by default, and it is still possible to overwrite the defaults
% using explicit options in \includegraphics[width, height, ...]{}
\setkeys{Gin}{width=\maxwidth,height=\maxheight,keepaspectratio}
\IfFileExists{parskip.sty}{%
\usepackage{parskip}
}{% else
\setlength{\parindent}{0pt}
\setlength{\parskip}{6pt plus 2pt minus 1pt}
}
\setlength{\emergencystretch}{3em}  % prevent overfull lines
\providecommand{\tightlist}{%
  \setlength{\itemsep}{0pt}\setlength{\parskip}{0pt}}
\setcounter{secnumdepth}{0}
% Redefines (sub)paragraphs to behave more like sections
\ifx\paragraph\undefined\else
\let\oldparagraph\paragraph
\renewcommand{\paragraph}[1]{\oldparagraph{#1}\mbox{}}
\fi
\ifx\subparagraph\undefined\else
\let\oldsubparagraph\subparagraph
\renewcommand{\subparagraph}[1]{\oldsubparagraph{#1}\mbox{}}
\fi

%%% Use protect on footnotes to avoid problems with footnotes in titles
\let\rmarkdownfootnote\footnote%
\def\footnote{\protect\rmarkdownfootnote}

%%% Change title format to be more compact
\usepackage{titling}

% Create subtitle command for use in maketitle
\providecommand{\subtitle}[1]{
  \posttitle{
    \begin{center}\large#1\end{center}
    }
}

\setlength{\droptitle}{-2em}

  \title{\vspace{0.5cm} \Large{Trust in political institutions and support for authoritarianism in Latin American students: Does civic knowledge make a difference?}}
    \pretitle{\vspace{\droptitle}\centering\huge}
  \posttitle{\par}
    \author{Daniel Miranda, Juan Carlos Castillo, Catalina Miranda \& José Daniel
Conejeros.}
    \preauthor{\centering\large\emph}
  \postauthor{\par}
      \predate{\centering\large\emph}
  \postdate{\par}
    \date{Borrador: 12/03/2020}

\usepackage[spanish,es-tabla]{babel}
\usepackage[utf8]{inputenc}
\usepackage{multicol}

\begin{document}
\maketitle
\begin{abstract}
Trust in political institutions represents a central component of
democratic systems. When citizens lack confidence in state bodies as the
government and the parliament, the legitimacy of democracy could be
challenged. To this regard, in the case of Latin America, we observe a
critical scenario, with a steady decrease of institutional trust in the
last decade, accompanied by alarming levels of support for authoritarian
regimes. Most of the evidence in this regards refers to adult
population, leaving sidelined the young generations who certainly have a
stake in the future of democracy in the region. This research analyzes
trust in political institutions and support for authoritarianism in
eighth-grade students, as well as how civic knowledge plays a role
enhancing or decreasing these political attitudes. The data corresponds
to the International Civic and Citizenship Education Study (ICCS) 2009
and 2016, in which six (Chile, Mexico, Colombia, Paraguay, Guatemala,
and the Dominican Republic) and five (Chile, Mexico, Colombia, Peru, and
the Dominican Republic) Latin American Countries participated,
respectively. Results indicate that civic knowledge has a double role.
On the one hand, students with higher levels of civic knowledge tend to
show lower levels of authoritarianism, while on the other, higher levels
of civic knowledge show lower levels of institutional trust. These
associations are pretty similar across time and countries, indicating a
particularity of the region. The consequences for democracy and
citizenship education are addressed in the conclusion.
\end{abstract}

\hypertarget{introduccion}{%
\section{Introducción}\label{introduccion}}

\break

\hypertarget{marco-teorico}{%
\section{Marco Teórico}\label{marco-teorico}}

\break

\hypertarget{metodologia}{%
\section{Metodología}\label{metodologia}}

\break

\hypertarget{resultados}{%
\section{Resultados}\label{resultados}}

\break

\hypertarget{discusion}{%
\section{Discusión}\label{discusion}}

\break

\hypertarget{conclusiones}{%
\section{Conclusiones}\label{conclusiones}}

\break

\hypertarget{referencias}{%
\section{Referencias}\label{referencias}}


\end{document}
